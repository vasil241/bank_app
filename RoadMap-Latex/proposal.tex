\documentclass[12pt, a4paper]{article}
\setlength{\oddsidemargin}{0.5cm}
\setlength{\evensidemargin}{0.5cm}
\setlength{\topmargin}{-1.6cm}
\setlength{\leftmargin}{0.5cm}
\setlength{\rightmargin}{0.5cm}
\setlength{\textheight}{24.00cm} 
\setlength{\textwidth}{15.00cm}
\parindent 0pt
\parskip 5pt
\pagestyle{plain}

\title{RoadMap of the Thesis}
\author{}
\date{}

\newcommand{\namelistlabel}[1]{\mbox{#1}\hfil}
\newenvironment{namelist}[1]{%1
\begin{list}{}
    {
        \let\makelabel\namelistlabel
        \settowidth{\labelwidth}{#1}
        \setlength{\leftmargin}{1.1\labelwidth}
    }
  }{%1
\end{list}}

\begin{document}
\maketitle

\begin{namelist}{xxxxxxxxxxxx}
\item[{\bf Title:}]
	Contract to Contract Calling in Algorand for a Use Case
\end{namelist}

\section*{Scope of Work}
\begin{itemize}
\item Provide a brief history of the issues to date.
\item Situate your particular topic within the broad area of research
\item Note that the field is changing, and more research is required on your topic.
\end{itemize}

\section*{Introduction} 

\subsection{Problem Statement}
\begin{itemize}
\item Identify a key point of concern (for example, increasing use or prominence, lack of research to date, response to an agenda, a new discovery, or perhaps one not yet applied to this context).

\item Refer to the literature only to the extent needed to demonstrate why your project is worth doing. Reserve your full review of existing theory or practice for later chapters.

\item Be sure that the motivation, or problem, suggests a need for further investigation.

\end{itemize}
\section*{Aim and Scope}
\begin{itemize}
\item Be sure that your aim responds logically to the problem statement.
\item Stick rigorously to a single aim. Do not include elements in it that describe
how you intend to achieve this aim; reserve these for a later chapter.
\item When you have written the conclusions to your whole study, check that they respond to this aim. If they don’t, change the aim or rethink your conclusions
\item If you change the aim, revise the motivation for studying it.
\item 	Be sure to establish the scope of your study by identifying limitations of fac- tors such as time, location, resources, or the established boundaries of particular fields or theories.
\end{itemize}
\begin{thebibliography}{9}
%\bibitem{knuth} D. E. Knuth. {\em The \TeX~book.}\/ Addison-Wesley,
%Reading, Massachusetts, 1984.
%\bibitem{lamport} L. Lamport. {\em \LaTeX~: A Document Preparation
%System}.\/ Addison-Wesley, Reading, Massachusetts, 1986.
%\bibitem{ken} Ken Wessen, Preparing a thesis using \LaTeX~, private
%communication, 1994.
%\bibitem{lamport2} L. Lamport. Document Production: Visual
%or Logical, {\em Notices of the Amer. Maths. Soc.},\/ Vol. 34,
%1987, pp. 621-624.
\end{thebibliography}


\end{document}

